% !TeX spellcheck = en_GB

\section{CLI Tool}

\subsection{Short Requirements Specification}
Use Cases i $\longrightarrow$ "Lastenheft".\\
Input and Output parameter identification.\\
Here is described what the finished tool must and can do.\\

\subsubsection{Exam Exercises}
An exam exercise consists out of a grammar, a word, a parsing table and a derivation tree. Creating a exam exercises must be possible. Therefore it is needed:
\begin{itemize}
	\item Selection of a possible exam exercise out of high scoring samples.
	\item Modifying of a possible exam exercise: Changing the grammar, changing the word, ?changing the pyramid (I think no, because of the strong interconnection between the grammar and the parsing table it is already covered through being able to change the grammar)?.
	\item Creation of a examExerciseInfo.txt file, that stores and updates pre calculated information for one sample while being modified. 
	\item Printable version of the finished exam exercise: grammar.png, parsingTable.png and derivationTable.png. Also the latex code that has been used for generating the .png-files should be stored for further modifications.
\end{itemize}

\subsubsection{Fun With CNF's and CYK}
Trying out stuff freestyle:
\begin{itemize}
	\item Se
	\item 
\end{itemize}

\pagebreak
\subsection{Overview - UML}

UML-Diagramm showing the general idea of the implementation.\\
List noteworthy used libraries here, too.\\
Maybe some information out of the statistics tool of IntelliJ.

\subsubsection{UML: More Detail 1}
\subsubsection{UML: More Detail 2}

\pagebreak
\subsection{User Interaction}

\noindent Here the specific must can do's are explained with short examples.
\subsubsection{Use Case 1}
\subsubsection{Use Case i}

\pagebreak