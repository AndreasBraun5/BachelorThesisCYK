% !TeX spellcheck = en_GB

\section{Course of Action}\label{CourseOfAction}

The informal goal is to find a suitable combination of a grammar and a word that meets the demands of an exam exercise. Also the CYK pyramid and one derivation tree of the word must be generated automatically as a "solution picture".\\
Firstly the exam exercise must have an upper limit of variables per cell while computing the CYK-pyramid.\\
Secondly the exam exercise must have one or more "special properties" so that it can be checked if the students have clearly understood the algorithm, e.g. "Excluding the possibility of luck."\\

\noindent The more formal goal is identify and determine parameters that in general can be used to define the properties of a grammar, so that the demanded restrictions are met. Also parameters could be identified for words, but "which is less likely to contribute, than the parameters of the grammar." [Second appointment with Martens] \\

\noindent Some introductory stuff:\\


\noindent Possible basic approaches for getting these parameters are the Rejection Sampling method and the "Tina+Wim" method. \\
Also backtracking plays some role, but right now I don't know where to put it. Backtracking is underapproach to Rejection Sampling. \\
Note: Starting with one half of a word and one half of a grammar.\\ 

\noindent Identify restrictions (=parameters) regarding the grammar.\\
Maybe find restrictions regarding the words, too.\\

\noindent Procedures for automated generation. Each generation procedure considers different restrictions and restriction combinations. The restrictions within one generation procedure can be optimized on its own. Up till now: \\
Generating grammars: DiceRolling, ...\\
Generating words: DiceRolling, ...\\

\noindent Parameter optimisation via theoretical and/or benchmarking approach. \\ Benchmarking = generate N grammars and test them, (N=100000). \\
Define a success rate and try to increase it. \\

\pagebreak

\noindent The overall strategy is as following: \\

\noindent1.) Identify theoretically a restriction/parameter for the grammar. Think about the influence it will have. Think also about correlations between the restrictions.\\
2.) Validate the theoretical conclusion with the benchmark. Test out the influence of this parameter upon the success rate. Try different parameter settings. \\

\noindent The ordering of step 1 and step 2 can be changed.\\

\noindent Maybe here could be used a data mining tool to help. \\

\pagebreak