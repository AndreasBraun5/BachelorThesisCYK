% !TeX spellcheck = en_GB

\section{CLI Tool}
Write much of this stuff in the appendix.
\subsection{Short Requirements Specification}
Use Cases i $\longrightarrow$ "Lastenheft".\\
Input and Output parameter identification.\\
Here is described what the finished tool must and can do.\\

\noindent Generating the latex code and storing it in .tex-file. Then converting the .tex-file to .pdf-file via:\\
Runtime rt = Runtime.getRuntime();\\
Process pr = rt.exec("pdflatex mydoc.tex");\\
Process pr = rt.exec("pdflatex mydoc.tex");\\
Process pr = rt.exec("pdflatex mydoc.tex");\\
The triple invocation of LaTeX is to ensure that all references have been properly resolved and any page layout changes due to inserting the references have been accounted for. [http://www.arakhne.org/autolatex/]

\subsubsection{Exam Exercises}
An exam exercise consists out of a grammar, a word, a parsing table and a derivation tree. Creating a exam exercises must be possible. Therefore it is needed:
\begin{itemize}
	\item Selection of a possible exam exercise out of high scoring samples $\longrightarrow$ calculateSamples.jar [Input parameter: countOfNewSamples (better scoring samples in exchange for longer computation time)], which upon execution fills samples.txt with new high scoring samples, together with its actual scoring model parameters. Out of this samples one can be selected manually that is used for an exam exercise.
	\item Modifying of a exam exercise candidate: Changing the grammar and changing the word. [?changing the pyramid (I think no, because of the strong interconnection between the grammar and the parsing table it is already covered through being able to change the grammar)?] $\longrightarrow$ calculateExamExercise.jar [Input parameter: examExercise.txt], that updates pre defined information for one sample upon execution.
	\item Predefined Information: It is a printable version of the finished exam exercise like grammar.png, parsingTable.png and derivationTable.png together with its latex code, that was used for its creation - modification later one possible. Also it is examExerciseInfo.txt, that has the information about its actual scoring model parameters. 
\end{itemize}

\subsubsection{Fun With CNF's and CYK}
Trying out stuff freestyle:
\begin{itemize}
	\item Se
	\item 
\end{itemize}

Maybe don't add the $Tree$ here. It is only needed as one part of the output of an $exercise$.
\begin{center}
	\resizebox{.6\linewidth}{!}{
		\begin{tikzpicture}[-,>=stealth',shorten >=1pt,auto,node distance=2cm,
		semithick]%,initial text={}]
		\tikzstyle{every state}=[fill=white,draw=black,text=black]
		
		\node     (I)            {$S1$};
		\node				(I') [below left of=I] {};
		\node (B) [left of=I']  {$A1$};
		\node (B1) [below of=B] {$c$};
		\node (C) [below right  of=I]       {$B6$};
		\node (D) [below right of=C]  {$B7$};
		\node (D2) [below left of=C]  {$C4$};
		\node (E) [below right of=D]  {$B8$};
		\node (F1) [below of=E]  {$b$};
		\node (E1) [below of=D]  {$C7$};
		\node (E3) [below of=D2]  {$C5$};
		\node (E2) [below left of=D2]  {$A2$};
		\node (F) [below  of=E1]  {$c$};
		\node (F3) [below of=E3]  {$A5$};
		\node (G1) [below  of=F3]  {$a$};
		\node (F2) [below right of=E3]  {$C6$};
		\node (G2) [below  of=F2]  {$c$};
		\node (H1) [below of=E2]  {$A3$};
		\node (H2) [below left of=E2]  {$B2$};
		\node (H3) [below left of=H2]  {$b$};
		\node (J1) [below of=H1]  {$A4$};
		\node (K1) [below of=J1]  {$a$};
		\node (J2) [below left of=H1]  {$B3$};
		\node (K3) [below of=J2]  {$b$};
		
		\path 
		(I) edge   node {$ $}   (B)
		(I) edge   node {$ $}   (C)
		(B) edge   node {$ $}   (B1)
		(C) edge   node {$ $}   (D)
		(D) edge   node {$ $}   (E)
		(C) edge   node {$ $}   (D2)
		(D2) edge   node {$ $}   (E3)
		(D) edge   node {$ $}   (E1)
		(D2) edge   node {$ $}   (E2)
		(E1) edge   node {$ $}   (F)
		(E) edge   node {$ $}   (F1)
		(E3) edge   node {$ $}   (F3)
		(E3) edge   node {$ $}   (F2)
		(F2) edge   node {$ $}   (G2)
		(F3) edge   node {$ $}   (G1)
		(E2) edge   node {$ $}   (H2)
		(E2) edge   node {$ $}   (H1)
		(H2) edge   node {$ $}   (H3)
		(H1) edge   node {$ $}   (J2)
		(H1) edge   node {$ $}   (J1)
		(J1) edge   node {$ $}   (K1)
		(J2) edge   node {$ $}   (K3)
		;
		
		\end{tikzpicture}
	}
	
\end{center}



\pagebreak
\subsection{Overview - UML}

UML-Diagramm showing the general idea of the implementation.\\
List noteworthy used libraries here, too.\\
Maybe some information out of the statistics tool of IntelliJ.

\subsubsection{UML: More Detail 1}
\subsubsection{UML: More Detail 2}

\pagebreak
\subsection{User Interaction}

\noindent Here the specific must can do's are explained with short examples.
\subsubsection{Use Case 1}
\subsubsection{Use Case i}

\pagebreak