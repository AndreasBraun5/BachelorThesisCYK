% das Papierformat zuerst
%\documentclass[a4paper, 11pt]{article}

% deutsche Silbentrennung
%\usepackage[ngerman]{babel}

% wegen deutschen Umlauten
%\usepackage[ansinew]{inputenc}

% hier beginnt das Dokument
%\begin{document}


\thispagestyle{empty}

%\begin{figure}[t]
% \includegraphics[width=0.6\textwidth]{abb/fh_koeln_logo}
%\end{figure}

\begin{figure}[t]
 \centering
 \includegraphics[width=0.6\textwidth]{abb/logo1}
~~~~~~~~~~
\end{figure}


\begin{verbatim}


\end{verbatim}

\begin{center}
\Large{University of Bayreuth}\\
\end{center}


\begin{center}
\Large{Institute for Computer Science}
\end{center}
\begin{verbatim}








\end{verbatim}
\begin{center}
\doublespacing
\textbf{\LARGE{\titleDocument}}\\
\singlespacing
\begin{verbatim}

\end{verbatim}
\textbf{{~\subjectDocument}}
\end{center}
\begin{verbatim}

\end{verbatim}
\begin{center}

\end{center}
\begin{verbatim}






\end{verbatim}
\begin{flushleft}
\begin{tabular}{llll}
\textbf{Topic:} & & The CYK-Algorithm, a CLI-Tool and different ways of \\ 
& & filling the Parsing Table & \\
& & \\
\textbf{Author:} & & Andreas Braun <www.github.com/AndreasBraun5>& \\
& & Matrikel-Nr. 1200197 & \\
& & \\
\textbf{Version date:} & & \today &\\
& & \\
\textbf{1. Supervisor:} & & Prof. Dr. Wim Martens &\\
\textbf{2. Supervisor:} & & M.Sc. Tina Trautner &\\
\end{tabular}
\end{flushleft}

\pagebreak
~
\pagebreak

\begin{verbatim}






















\end{verbatim}

\begin{center}
	To my parents.
\end{center}
