% !TeX spellcheck = en_GB
\section*{Abstract}\label{abtract}
Every year, lecturer in the field of theoretical computer science or an related one face the task to create an exam exercise that tests if their students have understood the way of working of the Cocke-Younger-Kasami algorithm. Various implementations and small online tools of the CYK algorithm can be found, but none actually assists during the process of creating an exercise.\\
Therefore various algorithms to generate specifically suitable exercises have been designed and compared through their success rates. The different approaches for these algorithms involve the uniform randomly distribution of elements and the general Bottom-Up and Top-Down parsing approaches.\\
A GUI tool to automatically generate these exam exercises has been implemented. Its functionality contains that input parameters such as the count of variables, the count of terminals and the size of the word can be given. Suitable exam exercises are generated and one can be chosen for further modification and creation of the final exam exercise.\\


~~

\section*{Zusammenfassung}\label{zusammenfassung}
Jedes Jahr stehen Dozenten der theoretischen Informatik oder eines verwandten Bereiches vor der Aufgabe Klausuraufgaben zu erstellen, die prüfen ob ihre Studenten die Arbeitsweise des Cocke-Younger-Kasami-Algorithmus verstanden haben. Verschiedene Implementierungen und kleinere Online-Tools des CYK-Algorithmus gibt es bereits, aber Keines unterstützt beim Prozess des Erstellen einer Aufgabe.\\
Verschiedene Algorithmen wurden zuerst entworfen, um genau passende Aufgaben zu generieren und wurden anschließend auch miteinander über ihre Erfolgsrate verglichen. Die unterschiedlichen Ansätze für die Algorithmen beinhalten das gleichmäßig zufällige Verteilen von Elementen und die allgemeinen Ansätze des Bottom-Up und Top-Down Parsings.\\
Es wurde ein GUI-Tool implementiert um automatisch Klausuraufgaben zu generieren. Die Funktionalität des Tools beinhaltet, dass Eingabewerte wie die Anzahl der Variablen, die Anzahl der Terminale und die Wortlänge gemacht werden können. Geeignete Klausuraufgaben werden automatisch generiert von denen Eine für weitere Modifikation und letztendlich für die Klausuraufgabenerstellung ausgewählt wird.\\



\pagebreak

