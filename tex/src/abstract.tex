% !TeX spellcheck = en_GB
\section*{Abstract}\label{abtract}
Every year, lecturers of theoretical computer science or a related field are confronted with the task of examining whether their students have understood the workings of the Cocke-Younger-Kasami algorithm. There are several implementations and smaller online tools of the CYK algorithm already in place, but none of them support the actual process of creating a suitable exercise for it.\\
Different algorithms were first designed to generate suitable exercises and then compared with each other using their success rate. The different approaches of these algorithms involve the uniformly randomly distribution of elements and the general Bottom-Up and Top-Down parsing approaches.\\
A GUI tool has been implemented to automatically generate exam exercises. The functionality of the tool includes that input parameters such as the number of variables, the number of terminals and the word length can be made. Suitable exam exercises are automatically generated from which one can be selected for further modification and creation of the exam exercise.\\

~~

\section*{Zusammenfassung}\label{zusammenfassung}
Jedes Jahr stehen Dozenten der theoretischen Informatik oder Doezenten eines verwandten Bereiches vor der Aufgabe Klausuraufgaben zu erstellen, um zu prüfen ob ihre Studenten die Arbeitsweise des Cocke-Younger-Kasami-Algorithmus verstanden haben. Verschiedene Implementierungen und kleinere Online-Tools des CYK-Algorithmus gibt es bereits, aber Keines unterstützt beim Prozess des Erstellens einer Aufgabe.\\
Verschiedene Algorithmen wurden zuerst entworfen, um genau passende Aufgaben zu generieren und wurden anschließend miteinander über ihre Erfolgsrate verglichen. Die unterschiedlichen Ansätze für die Algorithmen beinhalten das gleichmäßig zufällige Verteilen von Elementen und die allgemeinen Ansätze des Bottom-Up und Top-Down Parsings.\\
Es wurde ein GUI-Tool implementiert um automatisch Klausuraufgaben zu generieren. Die Funktionalität des Tools beinhaltet, dass Eingabewerte wie die Anzahl der Variablen, die Anzahl der Terminale und die Wortlänge gemacht werden können. Geeignete Klausuraufgaben werden automatisch generiert von denen Eine für weitere Modifikation und für die Klausuraufgabenerstellung ausgewählt werden kann.\\



\pagebreak

