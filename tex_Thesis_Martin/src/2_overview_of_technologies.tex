% !TeX spellcheck = en_GB

\section{Overview of technologies}\label{Overview}
Before we can go into detail about how to work with Hibernate Search in a generic environment, we will give a short overview of the relevant technologies first. We will explain why ORMs in general and the JPA specification in particular are beneficial. Then, we will explain what fulltext search engines are used for and give a short overview about the available solutions for Java. We will see that generalizing Hibernate Search for any JPA implementation is a good approach and that it has benefits over using the different search solutions available.

\subsection{Object Relational Mappers}
Nowadays, many popular languages like Java or C\# are object oriented. While SQL solutions for querying relational databases exist for these languages (JDBC for Java\footnote{Oracle JDBC overview, see~\cite{jdbc_oracle}}, OleDb for C\#\footnote{OleDb usage page, see~\cite{oledb_ms}}), the user either has to work with the rowsets manually or convert them into custom data transfer objects (DTO) to gain at least some "real" objects to work with. Both approaches don't suit the object oriented paradigm well as SQL "flattens" the data into rows when querying while a well designed class model would work with multiple classes in a hierarchy.
\\
\lstset{language=sql}
\begin{lstlisting}[frame=htrbl, caption={SQL "flattening" the author and book table into rows}, label={lst:flattening.sql}]
SELECT author.id, author.name, book.id, book.name 
FROM author_book, author, book
WHERE author_book.bookid = book.id
AND author_book.authorid = author.id
\end{lstlisting}
This is one of the points where Object Relational Mappers (ORM) come into use. They map tables to entity-classes and
enable users to write queries against these classes instead of tables. The returned objects are part of a object hierarchy and are easier to use from a object oriented point of view as even relations that were not included in a join can generally be re-queried automatically when needed.
\\
\lstset{language=java}
\begin{lstlisting}[frame=htrbl, caption={ORM query example}, label={lst:flattening.sql}]
List<Author> data = orm.query("SELECT a FROM Author a");
for(Author author : data) {
	// we can still fetch the books without joining in the query
	System.out.println("name: " + author.getName() + 
		", books: " + author.getBooks());
}
\end{lstlisting}
\noindent
This is especially useful if used in big software products as not all programmers have to know the exact details of the underlying database. The database system could even be completely replaced by another (provided the ORM supports the specific RDBMS), while the business logic would not change a bit.

\subsection{JPA}

The first version of the JPA standard was released in May 2006. From then on it rose to being probably the most commonly used persistence API for Java and is considered the "industry standard approach for Object Relational Mapping"\footnote{Wikibooks on Java Persistence, see~\cite{wikibooks_on_jpa}} \footnote{Stackoverflow JPA tag, see~\cite{stackoverflow_jpa}}. While mostly known for standardizing relational database mappers (ORM), it also supports other concepts like NoSQL\footnote{Hibernate OGM project homepage, see~\cite{hibernate_ogm}} \footnote{EclipseLink project homepage, see~\cite{eclipselink}} or XML storage\footnote{EclipseLink project homepage, see~\cite{eclipselink}}. However, when talking about JPA in this thesis, we will be focusing on the relational aspects of it. Currently, the newest version of this standard is 2.1 \footnote{JSR 338: JPA 2.1 specification, see~\cite{jpa_21_jcp}}.
\\\\
Some popular relational implementations are:
\begin{itemize}
	\item Hibernate ORM (Red Hat)\footnote{Hibernate ORM project homepage, see~\cite{hibernate_orm}}
	\item EclipseLink (Eclipse foundation)\footnote{EclipseLink project homepage, see~\cite{eclipselink}}
	\item OpenJPA (Apache foundation)\footnote{OpenJPA project homepage, see~\cite{openjpa}}
\end{itemize}
~\\
Using the standardized JPA API over any native ORM API has one really interesting benefit:
The specific JPA implementation can be swapped out as it comes with standards for many common use cases.
\\\\
This is particularily important if you are working in a Java EE environment. Java EE itself is a specification for platforms, mostly Web-servers (JPA is part of the Java EE spec).\footnote{Java EE specification on oracle.com~\cite{java_ee_spec}} Many Java EE Web-servers ship with a bundled JPA implementation that they are optimized for (WildFly with Hibernate ORM, GlassFish with EclipseLink, ...). This means that if the server is switched, it could also be a reasonable idea to swap out the JPA implementor. If everything in the application is written in a JPA compliant way, the user will then generally not encounter many problems related to this switch.

\pagebreak

\subsection{Fulltext search engines}

Conventional relational databases are good at retrieving and querying structured data. But if one wants to build a search engine atop a domain model, most RDBMS will only support the SQL-LIKE operator \footnote{w3schools on SQL LIKE, see~\cite{sql_like_w3schools}}:\\

\lstset{language=sql}
\begin{lstlisting}[frame=htrbl, caption={SQL LIKE operator in use}, label={lst:result2}]
SELECT book.id, book.name FROM book WHERE book.name LIKE %name%;
\end{lstlisting}
While this might be enough for some applications, this wildcard query doesn't support features a good search engine would need, for example:

\begin{itemize}
	\item fuzzy queries (variations of the original string will get matched, too)
	\item phrase queries (search for a specified phrase)
	\item regular expression queries (matches are determined by a regular expression)
	\item stemming and language specific optimisations
	\item comprehensive synonym support
\end{itemize}
There may exist some RDBMS that support similar query-types, but in the context of using an ORM we would then lose the ability to switch databases because of the usage of vendor-specific features that not every RDBMS supports.
\\\\
Fulltext search engines can be used to complement databases in this regard. They are generally not intended to be replacing the database, but add additional functionality by indexing the data that is to be searched in a more sophisticated way. We will now take a look at some of the most popular available options for Java developers (including Hibernate Search) focusing on their usage and features. After that, we will give the reasoning behind why a \textbf{generic} Hibernate Search is preferable to the other solutions.

\pagebreak

\subsubsection{Lucene}

\begin{quote}
	Apache Lucene™ is a high-performance, full-featured text search engine library written entirely in Java. It is a technology suitable for nearly any application that requires full-text search, especially cross-platform.\footnote{official Lucene website, see~\cite{lucene_apache_org}}
\end{quote}
Lucene serves as the basis for many fulltext search engines written in Java. It has many different utilities and modules aimed at search engine developers. However, it can be used on its own as well. Its latest stable version as of now is 5.3.0 \footnote{official Lucene website, see~\cite{lucene_apache_org}}.

\paragraph{Concepts}

As Lucene's focus is not on storing relational data, it comes with its own set of concepts. Following is a short overview of the most important ones. These are not only the basis for Lucene, but also for the other search engines we will discuss next, as they are based on Lucene's rich set of features.

\subparagraph{Index structure}
Lucene uses an \textbf{inverted index} to store data. This means that instead of storing texts mapped to the words contained in them, it works the other way around. All different words (terms) are mapped to the texts they occur in\footnote{Lucene basic concepts, see~\cite{lucene_basic_concepts}}, so it can be compared to a \(Map<String, List<Text>>\) in Java. Before anything can be searched using Lucene, it has to be added to the the index (indexed) first.

\subparagraph{Documents}
Documents are the data-structure Lucene stores and retrieves from the index. An index can contain zero or more Documents.

\subparagraph{Fields}
A Document consists of at least one field. Fields are basically tuples of key and value. They can be stored (retrievable from the index) and/or indexed (used for searches and generating hits).

\subparagraph{Analyzers}
Before documents get indexed, their fields are analysed with one of the many Analyzers first. Analysis is the process of modifying the input in a manner such that it can be searched upon (stemming, tokenization, ...).

\pagebreak

\subparagraph{Example index}
The following figure \ref{inverted_index} shows how an inverted index schematically looks like in Lucene. On the left we can see three different documents containing an id and the two text fields "field1" and "field2". The inverted index that stores references to these documents can be seen on the right. It contains all the different terms (field \& value) mapped to the id of the texts they are contained in. The values of these terms have been analysed before they were stored into the index as they only contain singular words instead of the original "sentences" from the left.
\\
\begin{figure}[ht]
	\centering
	\begin{minipage}{0.4\textwidth} 
		\centering
		\begin{tabular}{|ccc|}
			\hline
			\multicolumn{3}{|c|}{Documents} \\
			\specialrule{.25em}{0em}{0em}
			id & field1 & field2 \\
			\specialrule{.2em}{0em}{0em}
			1 & \specialcell{fulltext search \\ lucene} & search \\
			
			2 & lucene search & java \\
		
			3 & fulltext java & fulltext lucene \\
			\hline
		\end{tabular}
	\end{minipage}
	% Auffüllen des Zwischenraums
	\hfill
	\begin{minipage}{0.4\textwidth}
		\centering
		\begin{tabular}{|ccc|}
			\hline
			\multicolumn{3}{|c|}{Inverted Index} \\
			\specialrule{.25em}{0em}{0em}
			\multicolumn{2}{|c}{Term} & Occurences \\
			\specialrule{.2em}{0em}{0em}
			Field & Value & \\
			\specialrule{.1em}{0em}{0em}
			field1 & fulltext & 1,3 \\
			
			field1 & search & 1,2 \\
			
			field1 & lucene & 1,2 \\
			
			field1 & java & 3 \\
			
			field2 & search & 1 \\
			
			field2 & java & 2 \\
			
			field2 & fulltext & 3 \\
			
			field2 & lucene & 3 \\
			\hline
		\end{tabular}
	\end{minipage}
	\caption{schematic inverted index}
	\label{inverted_index}
\end{figure}

\pagebreak

\paragraph{Usage}
Using Lucene as a standalone engine requires the programmer to design the engine from the bottom up. The developer has to write all the logic, starting with the actual indexing code through to the code managing access to the index. The conversion from Java objects to Documents (for indexing) and back (for searching) have to be implemented as well. This whole process requires a lot of code to be written and the API only helps by providing the necessary tools. This is particularly problematic as the Lucene API tends to change a lot between versions and the code has to be kept up-to-date. It's not uncommon that whole features that were state-of-the-art in one version, are deprecated (potentially unstable, marked to be removed in the future) in the next release, resulting in big code changes being potentially necessary.

\paragraph{Features}
Lucene probably is the most complete toolbox to build a search-engine from. It has pre-built analyzers for many languages, a queryparser to support generating queries out of user input, a phonetic module, a faceting module, and many other features. While mostly known for its fulltext capabilities, it also has modules used for other purposes, for example the spatial module that enables geo-location query support.
\\\\
One benefit of its low-level API is that it can easily be extended with custom analyzers, query-types, etc, though. This is especially useful for more sophisticated search engines.

\pagebreak

\subsubsection{Fulltext search servers: ElasticSearch and Solr}
Lucene is the basis for two of the most popular search servers available: ElasticSearch (by elastic)\footnote{ElasticSearch Homepage, see~\cite{elasticsearch_homepage}} and Solr (sister project of Lucene)\footnote{Solr Homepage, see~\cite{solr_homepage}}. Their current stable versions are 1.7.1 \footnote{ElasticSearch Download website, see~\cite{elasticsearch_downloads_website}} and 5.3.0 \footnote{Solr Homepage, see~\cite{solr_homepage}} respectively.

\paragraph{Usage}
As both ElasticSearch and Solr are standalone server applications, they have to be configured before they can be used similar to the process of setting up a RDBMS. As they don't ship with any authentication mechanism by default they also have to be secured before they are used in production \footnote{Solr security, see~\cite{solr_security}} \footnote{elastic Shield (security for ElasticSearch), see~\cite{elasticsearch_security}}. Index changes and queries are done via a REST-like API (among other options).

\paragraph{Features}
As ElasticSearch and Solr are built upon Lucene, they support the same basic features that Lucene does, but add additional indexing and searching functionality and come with their own stack of tools to ease their usage (index inspectors, load analyzers, ... \footnote{Solr Administration (Core Specific Tools), see~\cite{solr_admin}} \footnote{ElasticHQ, see~\cite{elasticsearch_admin}}). They are generally used because of their good clustering capabilities (distribution \& replication) and are optimized for high throughput and scalability \footnote{ElasticSearch: Life inside a cluster, see~\cite{elasticsearch_clustering}} \footnote{Solr: Introduction to Scaling and Distribution, see~\cite{solr_clustering}}. As they are not running inside the client application (as a native Lucene implementation would) these kind of servers don't force the user to use a specific programming language (in our case a JVM based one like Java).

\pagebreak

\subsubsection{Hibernate Search}

From the GitHub README of Hibernate Search:
\begin{quote}
Full text search engines like Apache Lucene are very powerful technologies to add efficient free text search capabilities to applications. However, Lucene suffers several mismatches when dealing with object domain models. Amongst other things indexes have to be kept up to date and mismatches between index structure and domain model as well as query mismatches have to be avoided.

Hibernate Search addresses these shortcomings - it indexes your domain model with the help of a few annotations, takes care of database/index synchronization and brings back regular [JPA] managed objects from free text queries. \footnote{Hibernate Search GitHub README, see~\cite{hsearch_source_code_git}}
\end{quote}
\noindent
Hibernate Search's current stable version is 5.4.0.Final which is based on Lucene 4.10.4 \footnote{hibernate-search-engine on mvnrepository.org, see~\cite{hibernate_search_engine_mvnrepository}}.

\paragraph{Usage}
Hibernate Search is used in the context of JPA compliant applications using Hibernate ORM. It can easily be used by adding it to the classpath and setting some configuration properties in the JPA persistence.xml. It integrates seamlessly with JPA interfaces. 

\paragraph{Features}
Similar to ElasticSearch and Solr, Hibernate Search is built upon Lucene and has similar features regarding indexing, searching and clustering but it is designed to be used in a JPA environment: it indexes JPA entities and the queries return them again.
\\\\
It is tightly coupled with Hibernate ORM: while an integration with JPA is existent, Hibernate Search doesn't allow other JPA implementations than Hibernate ORM to be used as it internally relies on its code.
\\\\
For future versions the Hibernate Search team is planning on adding ElasticSearch and Solr as additional backends \footnote{Hibernate Search roadmap, see~\cite{hibernate_search_roadmap}} besides the already existing Lucene based backend and the optional Infinispan integration.

\pagebreak

\subsubsection{Why a generic Hibernate Search?}

For Hibernate ORM developers Hibernate Search is probably currently the easiest way to have fulltext search capabilities in their application. While the native Lucene backend might not be the perfect choice for some applications (because they want to share the index with applications written in e.g. C\#), the planned ElasticSearch and Solr backends would make up for this in the future.
\\\\
Developers using other JPA implementations like EclipseLink or OpenJPA currently don't have the option to use a similar API to Hibernate Search as the Compass project has been discontinued (last version: 2.2.0 from Apr 06, 2009 as of mvnrepository.org \footnote{see \url{http://mvnrepository.com/artifact/org.compass-project/compass/2.2.0}}).
\\\\
In order to create a fulltext engine integrated with generic JPA creating a separate solution similar to Hibernate Search wouldn't be beneficial as it would include a lot of work and would probably not get much recognition.
\\\\
A generic version of Hibernate Search however would use (most of) the already existing interfaces and would require a lot less code for the same behaviour and features as nearly all of the important Lucene logic can be found in modules not having any notion of Hibernate ORM. In fact, the only module of Hibernate Search requiring Hibernate ORM is "hibernate-search-orm".
\\\\
Ultimately this generic version of Hibernate Search could also inspire some remodelling of the original Hibernate Search to incorporate generic JPA, which could make Hibernate Search the de-facto standard for fulltext search for the complete JPA world.
\\\\
Using Hibernate Search and turning it into a general standard is definitely better than writing everything from scratch and thus "reinventing the wheel". \label{reinvent_the_wheel}

\pagebreak